\documentclass{article}

\title{Project Description}
\author{Sander Hellesø}

\usepackage{pdfpages}
\usepackage{titlesec}
\usepackage{subfig}
\usepackage{changepage}
\usepackage{graphicx}
\graphicspath{ {./assets/} }
\usepackage{geometry}
 \geometry{
 a4paper,
 total={120mm,225mm},
 left=45mm,
 top=45mm,
 }


\titleformat*{\section}{\LARGE\bfseries}
\titleformat*{\subsection}{\Large\bfseries}
\titleformat*{\subsubsection}{\large\bfseries}
\titleformat*{\paragraph}{\large\bfseries}
\titleformat*{\subparagraph}{\large\bfseries}
\setcounter{secnumdepth}{0}

\begin{document}
  \includepdf[fitpaper=true, pages=-]{./assets/cover.pdf}
  \setcounter{page}{1}
  
  \section{Description}
    \paragraph{}
    IIFYM App (Not offical name) is an application that will help users stick to their diet and acheive their fitness and health goals in a fun and motivating way.
    The application will heavily focus on the motivational aspect, utilizing visual charts, a streak system, and if time, friends feed, while
    at the same time provide the user with contiunal help on their fitness journey. The main feature of the app is that the user can daily enter 
    the calories and macronutriens eaten, the app will use these to continuously update the calorie requirement to user needs to reach their
    fitness goals.
    
    \hfill \break

  \section{Specification}
    \paragraph{}
    The following is the features that should be integrated into the application to ensure it provides the promised features.

      \subsection{Must}
        \begin{itemize}
          \item Registration / Account creation
          \item Login / Logout
          \item Profile Setup
          \item Daily input of calories eaten
          \item System to continuously calculate calorie requirement
          \item History of inputs / weekly / monthly / yearly
          \item Visualy see progress using charts
          \item Streak system / badges (logged in X days in a row etc)
        \end{itemize}

        \hfill \break

      \subsection{Should}
        \begin{itemize}
          \item Optional image upload to further add to progress section
          \item Some sort of social network (feed, friends, messages)
        \end{itemize}

        \hfill \break

      \subsection{Can}
        \begin{itemize}
          \item Import calories and macronutriens from datasets / other apps
        \end{itemize}

  \newpage
  \section{Illustrations}
    \paragraph{}
    On the page below, three different mockups are attatched. Home, progress and goals. These three will be the main sections of the application.

    \hfill \break

    \subsection{Home}
      \paragraph{}
      The home screen will show an overview of the users goal and an estimate of how long it is until the user reaches the goal.
      It also provides the user with the overview of the calories and macronutriens required to hit on the daily basis. An edit goal button
      is also provided such that the user easily can edit the goal from the main entry point of the app; the home screen.

      \hfill \break

    \subsection{Progress}
      \paragraph{}
      The progress screen is responsible for keeping track of the users progress and proivides a hisotry list, displatying
      information about each week tracked. The user should be able to further see details of the week by clicking on the specific week.
      The user is also provided with a checkmark or an cross, depending if the user managed to fill out every day that week.
     
      \hfill \break

    \subsection{Goals}
      \paragraph{}
      The goals screen will provide the user with ways to create / edit and see the current goal in details. 
      If the user dont have a goal (first time user), the app wil make them create a new goal, else the current goal can be modified
      or even deleted, the user must however always have one active goal in order for the app to work as planned.
      
      \hfill \break

    \subsection{Navigation}
      \paragraph{}
      Navigation between screens are mainly done by utilizing a fixed bottombar component, available on every screen, highlightng the current
      screen. This ensures the user easily can toggle sceens comfortably due to the natural position of the tabbar (easy access from thumbs). 
      There should alse be a topbar providing a back button if the screens are nested, eg user is watching details of week 4, week 4 is under progress history.

  \begin{adjustwidth}{0px}{150px}
    \begin{figure}
      \begin{tabular}{cc}
        \includegraphics[scale=0.4]{home} & \includegraphics[scale=0.4]{progress} \\
      (a) Home & (b) Progress \\[25pt]
      \includegraphics[scale=0.4]{goals}\\
      (c) Goals \\[25pt]
      \end{tabular}
      \caption{Mocups of app screens}
    \end{figure}
  \end{adjustwidth}


      
\end{document}