\documentclass{article}

\title{Project Report}
\author{Sander Hellesø}

\usepackage{pdfpages}
\usepackage{titlesec}
\usepackage{subfig}
\usepackage{changepage}
\usepackage{graphicx}
\graphicspath{ {./assets/} }
\usepackage{geometry}
 \geometry{
 a4paper,
 total={120mm,225mm},
 left=45mm,
 top=45mm,
 }


\titleformat*{\section}{\LARGE\bfseries}
\titleformat*{\subsection}{\Large\bfseries}
\titleformat*{\subsubsection}{\large\bfseries}
\titleformat*{\paragraph}{\large\bfseries}
\titleformat*{\subparagraph}{\large\bfseries}
\setcounter{secnumdepth}{0}

\begin{document}
\includepdf[fitpaper=true, pages=-]{./assets/cover3.pdf}
\setcounter{page}{1}

\tableofcontents
\newpage

\section{What it is}
\paragraph{}
    IIFYM App (Not offical name) is an application that will help users stick to their diet and acheive their fitness and health
    goals in a fun and motivating way. The application heavily focuses on the motivational aspect, utilizing visual charts and progression schemas, while
    at the same time provide the user with contiunal help on their fitness journey. The application make it easy to lose or gain weight without having to know
    much about the details going on in the background.

    \hfill \break
    \hfill \break
    \hfill \break

    \begin{figure}[h!]
        \includegraphics[scale=0.20]{auth}
        \centering
        \caption{Authentication Screen}
    \end{figure}


\newpage

\section{Why do we need it?}
\paragraph{}
    In todays society, there is alot of focus on health and fitness. When a topic is popular, there is alot of information, but often most of it is overcomplicated or outdated.
    IIFYM (If it fits your macros) is a popular dieting method that allows you to eat whatever you want as long as the calories and macronutrients are tracked and not exceeded.
    Since this method is a lot less restrictive that the standard "meal plan", it allows you the flexibility to eat your favorite foods while still seeing results.

\paragraph{}
    This app simplifies this for the user. Allowing the user to very easy calculate their calore requirements just by entering a few facts about themself. The only thing the
    user then has to do is to simply log their weight whenever they can. The app will recalculate the users calorie and macronutrients need and show the progression aswell
    as the estimated weeks left until the goal is reached. It cant become any simplier than this.

    \hfill \break
    \hfill \break

    \begin{figure}[h!]
        \includegraphics[scale=0.45]{iifym}
        \centering
        \caption{Example of different foods with the same amount of fat}
    \end{figure}

\newpage

\section{User Guide}
\paragraph{}
    A simple user guide to get started with the application.

    \hfill \break

\subsection{Authenticating}
\paragraph{}
    It is very simple to get started. If you dont have an account yet, you can create one on the first screen you see when opening the app
    By clicking on the white bordered "Sign Up" button. If you happend to have an account from before, you can log in with that account
    using the highlighted purple button "Login".

    \hfill \break
    \hfill \break
    
    \begin{figure}[h!]
        \includegraphics[scale=0.20]{auth}
        \centering
        \caption{Authentication screen where the user can either log in or sign up}
    \end{figure}
    \newpage

\subsection{Creating a profile}
\paragraph{}
    Once auhtenticated you need to create a profile if you haven't yet. The profile setup asks you what your height is, your gender aswell as your birthday.
    This data is used in the formulas required to calculate the your calorie and macronutrients requirements.

\paragraph{}
    To set your height, simply use your finger and slide the circle to whatever your height is. Selecting the gender can be done by clicking on the "Select a gender" field,
    a dropdown containing the options "male" and "female" appears, select whatever is correctly identifiable to you. Lastly, select your birthday by clicking on the "Select date of birth"
    field, a popup with a interactive calendar appears. Use this calendar to select your birthday in year, month and date. Once done, you can finish the setup by clicking the big, purple
    button "Save and continue to goal".

    \hfill \break
    \hfill \break

    \begin{figure}[h!]
        \includegraphics[scale=0.20]{profile}
        \centering
        \caption{Profile setup screen}
    \end{figure}
    \newpage

\subsection{Entering your weight and goal}
\paragraph{}
    In order to create your first goal, we need to set the goal! Simply set your current weight by using the slider "My current weight", this will serve as the start
    of your tracking journey. The only thing left is to select your goal weight. Depending on if you want to gain or lose weight, the calories and macronutrients will differ.
    Select the goal weight by using the slider "My goal weight". Once you are happy with both values, continue to the next step by clicking the big, purple
    button "Continue to next step".

    \hfill \break
    \hfill \break

    \begin{figure}[h!]
        \includegraphics[scale=0.20]{stats}
        \centering
        \caption{Selecting your weight screen}
    \end{figure}
    \newpage

\subsection{Setting the daily activity}
\paragraph{}
    The next thing we need is to now about your daily activity. Simply select the most appropriate option out of the following by clicking on them:

    \begin{itemize}
        \item Sedentary
        \item Light Active
        \item Active
        \item Very Active
    \end{itemize}

\paragraph{}
    Note that the options above is your lifestyle WITHOUT workouts included. Be honest with yourself about what you do outside of the gym (work, trips, etc).
    Once you are happy with your selected option, continue to the next step by clicking the big, purple
    button "Continue to next step".
   
    \hfill \break
    
    \begin{figure}[h!]
        \includegraphics[scale=0.20]{activity}
        \centering
        \caption{Selecting your activity screen}
    \end{figure}
    \newpage

\subsection{Workout frequency}
\paragraph{}
    The final step in creating your first goal is to determine the workout frequency you will be doing. Start by selecting the amount of workouts you will be doing
    per week by using the slider labeled by "Workouts per week". The amount is limited to 7 as we only require one hard session a day to determine the calorie and 
    macronutrients requirements. We also need to know about the length of each session. This is a rough estimate in minutes. Select this amount by using the 
    "Minutes per workout" slider. Finally, select the intensity of each workout by clicking on the most appropriate tab out of the options listed. 
    Finish the setup by clicking the big, purple button "Create goal".
  
    \hfill \break
    \hfill \break
        
    \begin{figure}[h!]
        \includegraphics[scale=0.20]{workouts}
        \centering
        \caption{Setting the workout frequency screen}
    \end{figure}
    \newpage

\subsection{Dashboard}
\paragraph{}
    Once the goal has been successfully setup. You can access the dashboard. This will now be the first thing you see when logging into the app.
    The dashboard contains information about the calories you need to eat every day, aswell as the macronutrients and its separation.

\paragraph{}
    The dashboard also includes a linegraph displaying your daily or weekly progression depending on the active toggle. Note that the data will
    only show if there actually is data to display.
  
    \hfill \break
    \hfill \break
        
    \begin{figure}[h!]
        \includegraphics[scale=0.20]{dashboard}
        \centering
        \caption{The app dashboard, displaying the data}
    \end{figure}
    \newpage

\subsection{Logging weight}
\paragraph{}
    You can log your daily weight by clicking the big, round pluss button located at the bottom of the dashboard. This button will open a new screen
    where you can enter todays weight and save it. Once a weight has been saved, you can see it displayed in the linegraph. 
    
\paragraph{}
    For every seveth (7) log entry.
    The application will recalculate your calorie and macronutrient requirements to keep your body in the required state to lose fat / gain muscle depending
    on your goal. You do not need to log EVERY day, but the more you do, the more accurate numbers we can provide which will eventually provide better results
    so it is recommended to log every day if you can.
      
    \hfill \break
    \hfill \break
            
    \begin{figure}[h!]
        \includegraphics[scale=0.20]{log}
        \centering
        \caption{Log your weight screen}
    \end{figure}
    \newpage

\subsection{Settings}
\paragraph{}
    The settings screen can be accessed by clicking on the grindwheel on the right side og the bottom navigation bar, located on the dashboard.
    From the settings screen you can either log out, or delete your account. If you decide to logout, you will be returned to the authentication screen
    and must re-authenticate in order to get back to the dashboard. If you decide to delete your accound, a prompt must be filled out by entering your password.
    If completed, all data will be wiped for the account and the user will also be returned to the authentication screen.
  
    \hfill \break
    \hfill \break
                
    \begin{figure}[h!]
        \includegraphics[scale=0.20]{settings}
        \centering
        \caption{The settings screen where you can logout or delete the account}
    \end{figure}

\newpage

\section{Design}
\paragraph{}
    Discussing the techinaclity behind the application aswell as the UI and UX.

    \hfill \break
    \hfill \break

\subsection{UI and UX}
\paragraph{}
    qqweqwe

    \hfill \break
    \hfill \break

\subsection{Technical}
\paragraph{}
    qqweqwe

    \hfill \break
    \hfill \break

\newpage

\section{The core Features}
\paragraph{}
    The following features is what binds the application together and should be weighted during grading:

    \hfill \break

    \begin{itemize}
        \item Homescreen, Easy access to both login and signup depending on the users need
        \item Profile and goal setup, Good UX that easily allows the user to use the app without thinking
        \item Startup logic / Splashscreen, Loads in data and set activity depending on where the user is (Signed out, no profile, no goal etc)
        \item Dashboard, MPCharts API
        \item Bottom Navigation UI and button
        \item Design and professional feel 
        \item Account deletion, A crucial step alot of apps are missing
    \end{itemize}
\newpage    


\newpage

\section{What could be done different}
\paragraph{}
    Due to the fact that I  had thought about the app idea for a while and its features. I came in to the development of the project with a clear idea of what to do.
    If I where to change anything, it would be to refactor the main logic of the math calculation out of the static User class and have an own dedicated math class
    for this job. 

\paragraph{}
    There could also potensially be improvements of the way I access the User object, a different solution could be to pass it along all intents and remove
    the static modifier, but on the otherhand, I feel that this greatly increases the complexity of the code. If there were more time I would definently add in some unit tests
    to ensure the logic ACTUALLY is correct, even though this has been checked manually (see Testing section).

\paragraph{}
    There is also one bug that occurs at very rare times, when the user
    logs their weight, the calories and macronutrients sometimes get reset to 0 and the user must log back to see the correct data. This is however a very rare occurence but should
    be taken note to. 

    \hfill \break
    \hfill \break

\section{Future Development}
\paragraph{}
    For the future I would love to add a tabbed menu option that navigates the user to a list where they can see details about each log entry aswell as the estimated
    weeks left to reach the goal. I also want to develop a button where the user can update their goal, this is something that can be done in a day due to the logic
    is all ready developed and tested.

\paragraph{}
    A cool feature I would love to add in the future would be a notification system that alerts the user if it has gone X hours since the last log to motivate the
    user and remind to log the weight for the day.

    \hfill \break
    \hfill \break

\section{Testing}
\paragraph{}
    No unit tests were written during the development of the application. The testing done was manual and every feature was tested this way to ensure that the "mistakes" of a human
    could potensially interfere with the logic.

\newpage

\section{Conclusion}
\paragraph{}
    To conculde the process of building the application, it was alot of fun working and learning about the Android environment and the cool API's it expose to us. 
    Most of the documentation was well documented and I had to main "roadblock" when it came to the development process. I feel that I managed to deliver each iteration
    of the application in good time and I am happy with the final outcome. The application does what it was designed to do and can be a great tool for the people that
    needs the help in regards to fitness and nutrition.
\newpage

\section{Resources and Help}
\paragraph{}
    The following resources has been used to develop features of the app:

    \begin{itemize}
        \item Firebase Docs (https://firebase.google.com/docs/android/setup)
        \item MPChart Docs (https://github.com/PhilJay/MPAndroidChart)
        \item Math Formulas (https://steelfitusa.com/2018/10/calculate-tdee/)
    \end{itemize}

\end{document}