\documentclass{article}

\title{Systemdesign}
\author{Sander Hellesø}

\usepackage{pdfpages}
\usepackage{titlesec}
\usepackage{subfig}
\usepackage{changepage}
\usepackage{graphicx}
\graphicspath{ {./assets/} }
\usepackage{geometry}
 \geometry{
 a4paper,
 total={120mm,225mm},
 left=45mm,
 top=45mm,
 }


\titleformat*{\section}{\LARGE\bfseries}
\titleformat*{\subsection}{\Large\bfseries}
\titleformat*{\subsubsection}{\large\bfseries}
\titleformat*{\paragraph}{\large\bfseries}
\titleformat*{\subparagraph}{\large\bfseries}
\setcounter{secnumdepth}{0}

\begin{document}
  \includepdf[fitpaper=true, pages=-]{./assets/cover2.pdf}
  \setcounter{page}{1}
  
  \section{Structure}
    \paragraph{}
    The application will be utlizing a combination of fragments and services. Services will be used when uploading photos and while saving the data to the database. 
    Fragments will be heavily used to activate the different acitivties of the app, however if we can get around with a more simple solution to navigate around activites,
    the need for fragments might be reduced.

    \hfill \break
    \hfill \break

    \subsection{Datastorage}
        \paragraph{}
        Firebase will be the platform of choice when it comes to datastorage. The data should be structured no longer than 2 childS deep than the parent tree to reduce the amount of read per query.
        There is no need for realtime updates so Firestore will most likely be utilized over the realtime database.

        \paragraph{}
        Data will mainly be provided by the user. There is some validation of data required, but most of this is already handled by the different components available to us. 
        Data that needs to be validated is mainly the numbers for the different nutrients, images uploaded might needs to be reduced in size if uploaded.

    \hfill \break
    \hfill \break
    
    \subsection{Trees}
        \paragraph{}
        Users record has the generated firebase ID as the parent, Goals and Weeks will use the releated users firebase ID.
        Goals will most likely be unqiue such that a user can only have one goal at a time. If we will store old goals for history
        is still not decided and need to be decided in the future. Weeks will contain week number, amount of days filled out, and a list of days, where each day
        contain data about the nutrients for that specific day, in the end summing up to a weekly total. Every week will also contain and optional field; Progress picture.
        This picture will either be in base64 or a reference to the uploaded files bucket if we decide to use the firebase service.
        On the next page, a list containing a rough idea on how the data will be structured is provided.

        \newpage
        \hfill \break
        \subsubsection{JSON Structure}

        \hfill \break

        \begin{itemize}
          \item Users
            \begin{itemize}
                \item Email
                \item Passwordhash
                \item Streak
            \end{itemize}
            \hfill \break
          \item Goals
            \begin{itemize}
                \item Current Weight
                \item Goal Weight
                \item Estimates weeks to goal
                \item Total Calories
                \item Fat
                \item Protein
                \item Carbohydrates
                \item Progress picture
            \end{itemize}
            \hfill \break
          \item Weeks
            \begin{itemize}
                \item Week number
                \item Days completed
                    \begin{itemize}
                        \item Day
                        \item Total Calories
                        \item Fat
                        \item Protein
                        \item Carbohydrates
                    \end{itemize}
            \end{itemize}
        \end{itemize}

    \newpage

    \section{Main Sections}
        \paragraph{}
        As written in the Project Description, the application will contain the following sections. 
        Note that Every section with exception of Signup / Login is hidden behind authentication. A user must have an account in order to continue to these.

        \hfill \break

        \begin{itemize}
            \item Signup / Login
            \item Home
            \item Goals
            \item Settings
            \item Progress
        \end{itemize}

        \hfill \break

        \subsection{Flow}
            \paragraph{}
            When first launching the app. The user will be provided with either the Home screen if authenticated, or the Signup / Login.
            Once authenticated the user will have access to Goals, Settings and Progress. The navigation between these are done with a fixed
            bottom navigation bar that is always available, no matter what screen the user is currently on. On nesten screens, a backbutton in the top left corner
            is provided. An example of a nested screen is when a user is on the Progress screen and wants to see details about a specific day. 
            The user will click on the specific day and be transitioned into a new screen containing the details about that specific day. From here,
            the user obviuously should be able to go back to the list of weeks, so a back button is provided in this scenario.

        \hfill \break

        \subsection{Design}
            \paragraph{}
            In regards to design, it will mainly be Material Design, utilizing the provided components given to us. 
            However, most of them will be customized to fit the desired theme and look I want to achive. 
            Clean colors, light shadows, bold text with a nice font, aswell as enough space is important for this application
            such that it feel professional and not just another basic Android App.  

    \newpage

    \section{Resources}
        \paragraph{}
        Most of the applications illustation resources will be available from Material, using the Material Icons set. If we were to implement the social newsfeed,
        a fitting sound might be needed and would thus be required to download from an external service with the correct license.

    \hfill \break

    \section{Classes}
        \paragraph{}
        Below is a list of the classes required to build the application at this time. This list will be updated as the application progresses thorugh development.

        \hfill \break

        \begin{itemize}
            \item ImageUpload       - Handles proccesing and uploading of taken image
            \item CalculateMacros   - Calculation of the users calories and macronutrients
            \item Week              - Contains methods for a specific week of a users goal
            \item Day               - Contains methods for a specific day of a users week goal
            \item User              - Contain info and methods for the current logged in user  
        \end{itemize}

        \hfill \break

    \section{Libaries}
        \paragraph{}
        Below is a list of the required libaries required to build the application. This list might be updated as the application progresses thorugh development.

        \hfill \break

        \begin{itemize}
            \item Firebase       - Cloud Database and authentication provided by Google
            \item Cloud Storage  - Cloud Storage service, a subfeature of Firebase
            \item MPAndroidChart - A powerful Android chart view / graph view library, supporting line- bar- pie- radar- bubble- and candlestick charts as well as scaling, dragging and animations.
        \end{itemize}

\end{document}