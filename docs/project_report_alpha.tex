\documentclass{article}

\title{Project Description}
\author{Sander Hellesø}

\usepackage{pdfpages}
\usepackage{titlesec}
\usepackage{subfig}
\usepackage{changepage}
\usepackage{graphicx}
\graphicspath{ {./assets/} }
\usepackage{geometry}
 \geometry{
 a4paper,
 total={120mm,225mm},
 left=45mm,
 top=45mm,
 }


\titleformat*{\section}{\LARGE\bfseries}
\titleformat*{\subsection}{\Large\bfseries}
\titleformat*{\subsubsection}{\large\bfseries}
\titleformat*{\paragraph}{\large\bfseries}
\titleformat*{\subparagraph}{\large\bfseries}
\setcounter{secnumdepth}{0}

\begin{document}
\includepdf[fitpaper=true, pages=-]{./assets/cover3.pdf}
\setcounter{page}{1}

\part{Project Description}

  
\section{Description}
  \paragraph{}
  IIFYM App (Not offical name) is an application that will help users stick to their diet and acheive their fitness and health goals in a fun and motivating way.
  The application will heavily focus on the motivational aspect, utilizing visual charts, a streak system, and if time, friends feed, while
  at the same time provide the user with contiunal help on their fitness journey. The main feature of the app is that the user can daily enter 
  the calories and macronutriens eaten, the app will use these to continuously update the calorie requirement to user needs to reach their
  fitness goals.
  
  \hfill \break

\section{Specification}
  \paragraph{}
  The following is the features that should be integrated into the application to ensure it provides the promised features.

    \subsection{Must}
      \begin{itemize}
        \item Registration / Account creation
        \item Login / Logout
        \item Profile Setup
        \item Daily input of calories eaten
        \item System to continuously calculate calorie requirement
        \item History of inputs / weekly / monthly / yearly
        \item Visualy see progress using charts
        \item Streak system / badges (logged in X days in a row etc)
      \end{itemize}

      \hfill \break

    \subsection{Should}
      \begin{itemize}
        \item Optional image upload to further add to progress section
        \item Some sort of social network (feed, friends, messages)
      \end{itemize}

      \hfill \break

    \subsection{Can}
      \begin{itemize}
        \item Import calories and macronutriens from datasets / other apps
      \end{itemize}

\newpage
\section{Illustrations}
  \paragraph{}
  On the page below, three different mockups are attatched. Home, progress and goals. These three will be the main sections of the application.

  \hfill \break

  \subsection{Home}
    \paragraph{}
    The home screen will show an overview of the users goal and an estimate of how long it is until the user reaches the goal.
    It also provides the user with the overview of the calories and macronutriens required to hit on the daily basis. An edit goal button
    is also provided such that the user easily can edit the goal from the main entry point of the app; the home screen.

    \hfill \break

  \subsection{Progress}
    \paragraph{}
    The progress screen is responsible for keeping track of the users progress and proivides a hisotry list, displatying
    information about each week tracked. The user should be able to further see details of the week by clicking on the specific week.
    The user is also provided with a checkmark or an cross, depending if the user managed to fill out every day that week.
   
    \hfill \break

  \subsection{Goals}
    \paragraph{}
    The goals screen will provide the user with ways to create / edit and see the current goal in details. 
    If the user dont have a goal (first time user), the app wil make them create a new goal, else the current goal can be modified
    or even deleted, the user must however always have one active goal in order for the app to work as planned.
    
    \hfill \break

  \subsection{Navigation}
    \paragraph{}
    Navigation between screens are mainly done by utilizing a fixed bottombar component, available on every screen, highlightng the current
    screen. This ensures the user easily can toggle sceens comfortably due to the natural position of the tabbar (easy access from thumbs). 
    There should alse be a topbar providing a back button if the screens are nested, eg user is watching details of week 4, week 4 is under progress history.

\begin{adjustwidth}{0px}{150px}
  \begin{figure}
    \begin{tabular}{cc}
      \includegraphics[scale=0.4]{home} & \includegraphics[scale=0.4]{progress} \\
    (a) Home & (b) Progress \\[25pt]
    \includegraphics[scale=0.4]{goals}\\
    (c) Goals \\[25pt]
    \end{tabular}
    \caption{Mocups of app screens}
  \end{figure}
\end{adjustwidth}

\part{System Design}
\section{Structure}
    \paragraph{}
    The application will be utlizing a combination of fragments and services. Services will be used when uploading photos and while saving the data to the database. 
    Fragments will be heavily used to activate the different acitivties of the app, however if we can get around with a more simple solution to navigate around activites,
    the need for fragments might be reduced.

    \hfill \break
    \hfill \break

    \subsection{Datastorage}
        \paragraph{}
        Firebase will be the platform of choice when it comes to datastorage. The data should be structured no longer than 2 childS deep than the parent tree to reduce the amount of read per query.
        There is no need for realtime updates so Firestore will most likely be utilized over the realtime database.

        \paragraph{}
        Data will mainly be provided by the user. There is some validation of data required, but most of this is already handled by the different components available to us. 
        Data that needs to be validated is mainly the numbers for the different nutrients, images uploaded might needs to be reduced in size if uploaded.

    \hfill \break
    \hfill \break
    
    \subsection{Trees}
        \paragraph{}
        Users record has the generated firebase ID as the parent, Goals and Weeks will use the releated users firebase ID.
        Goals will most likely be unqiue such that a user can only have one goal at a time. If we will store old goals for history
        is still not decided and need to be decided in the future. Weeks will contain week number, amount of days filled out, and a list of days, where each day
        contain data about the nutrients for that specific day, in the end summing up to a weekly total. Every week will also contain and optional field; Progress picture.
        This picture will either be in base64 or a reference to the uploaded files bucket if we decide to use the firebase service.
        On the next page, a list containing a rough idea on how the data will be structured is provided.

        \newpage
        \hfill \break
        \subsubsection{JSON Structure}

        \hfill \break

        \begin{itemize}
          \item Users
            \begin{itemize}
                \item Email
                \item Passwordhash
                \item Streak
            \end{itemize}
            \hfill \break
          \item Goals
            \begin{itemize}
                \item Current Weight
                \item Goal Weight
                \item Estimates weeks to goal
                \item Total Calories
                \item Fat
                \item Protein
                \item Carbohydrates
                \item Progress picture
            \end{itemize}
            \hfill \break
          \item Weeks
            \begin{itemize}
                \item Week number
                \item Days completed
                    \begin{itemize}
                        \item Day
                        \item Total Calories
                        \item Fat
                        \item Protein
                        \item Carbohydrates
                    \end{itemize}
            \end{itemize}
        \end{itemize}

    \newpage

    \section{Main Sections}
        \paragraph{}
        As written in the Project Description, the application will contain the following sections. 
        Note that Every section with exception of Signup / Login is hidden behind authentication. A user must have an account in order to continue to these.

        \hfill \break

        \begin{itemize}
            \item Signup / Login
            \item Home
            \item Goals
            \item Settings
            \item Progress
        \end{itemize}

        \hfill \break

        \subsection{Flow}
            \paragraph{}
            When first launching the app. The user will be provided with either the Home screen if authenticated, or the Signup / Login.
            Once authenticated the user will have access to Goals, Settings and Progress. The navigation between these are done with a fixed
            bottom navigation bar that is always available, no matter what screen the user is currently on. On nesten screens, a backbutton in the top left corner
            is provided. An example of a nested screen is when a user is on the Progress screen and wants to see details about a specific day. 
            The user will click on the specific day and be transitioned into a new screen containing the details about that specific day. From here,
            the user obviuously should be able to go back to the list of weeks, so a back button is provided in this scenario.

        \hfill \break

        \subsection{Design}
            \paragraph{}
            In regards to design, it will mainly be Material Design, utilizing the provided components given to us. 
            However, most of them will be customized to fit the desired theme and look I want to achive. 
            Clean colors, light shadows, bold text with a nice font, aswell as enough space is important for this application
            such that it feel professional and not just another basic Android App.  

    \newpage

    \section{Resources}
        \paragraph{}
        Most of the applications illustation resources will be available from Material, using the Material Icons set. If we were to implement the social newsfeed,
        a fitting sound might be needed and would thus be required to download from an external service with the correct license.

    \hfill \break

    \section{Classes}
        \paragraph{}
        Below is a list of the classes required to build the application at this time. This list will be updated as the application progresses thorugh development.

        \hfill \break

        \begin{itemize}
            \item ImageUpload       - Handles proccesing and uploading of taken image
            \item CalculateMacros   - Calculation of the users calories and macronutrients
            \item Week              - Contains methods for a specific week of a users goal
            \item Day               - Contains methods for a specific day of a users week goal
            \item User              - Contain info and methods for the current logged in user  
        \end{itemize}

        \hfill \break

    \section{Libaries}
        \paragraph{}
        Below is a list of the required libaries required to build the application. This list might be updated as the application progresses thorugh development.

        \hfill \break

        \begin{itemize}
            \item Firebase       - Cloud Database and authentication provided by Google
            \item Cloud Storage  - Cloud Storage service, a subfeature of Firebase
            \item MPAndroidChart - A powerful Android chart view / graph view library, supporting line- bar- pie- radar- bubble- and candlestick charts as well as scaling, dragging and animations.
        \end{itemize}
    
    \newpage

    \part{Alpha}
        \paragraph{}
        The alpha release contains several of the core features required for the app to function. Note that the googe-service.json file is NOT included. You must provide your own and place it at root level.

    \hfill \break

    \section{Feature List}
        \paragraph{}
        Bellow is a list containg the features implemented into the alpha release.

        \hfill \break

        \begin{itemize}
            \item Splash Screen with intent selector depending on user state
            \item Database connection to Firebase
            \item Sign up to application
            \item Authenticate and log in
            \item Create user profile
            \item Create goals
            \item Calculate calorie requirements
            \item Calculate macro divisions (Fat, Protein, Carbohydrats)
            \item Estimate amount of weeks required to reach goal
        \end{itemize}
    
    \hfill \break

    \section{Screenshots}
        \paragraph{}
        The upcoming pages includes screenshots of the different implemented screens.
        
    \addtolength{\topmargin}{-.575in}
    \begin{adjustwidth}{0px}{0px}
        \begin{figure}
          \begin{tabular}{cc}
            \includegraphics[scale=0.15]{splash} & \includegraphics[scale=0.15]{auth} \\
          (a) Splash & (b) Authenticate \\[25pt]
          \includegraphics[scale=0.15]{signup} & \includegraphics[scale=0.15]{login} \\
          (c) Sign Up & (d) Log In \\[25pt]
          \end{tabular}
        \end{figure}
      \end{adjustwidth}

      \begin{adjustwidth}{0px}{0px}
        \begin{figure}
          \begin{tabular}{cc}
            \includegraphics[scale=0.15]{profile} & \includegraphics[scale=0.15]{stats} \\
          (e) Create Profile & (f) Choose goal stats \\[25pt]
          \includegraphics[scale=0.15]{activity} & \includegraphics[scale=0.15]{workouts} \\
          (g) Choose daily actvity & (h) Choose workout frequency \\[25pt]
          \end{tabular}
        \end{figure}
      \end{adjustwidth}

      \begin{adjustwidth}{0px}{0px}
        \begin{figure}
          \begin{tabular}{cc}
            \includegraphics[scale=0.15]{home2} & \includegraphics[scale=0.15]{date} \\
          (i) Home after setup & (j) Birthday picker \\[25pt]
          \end{tabular}
        \end{figure}
      \end{adjustwidth}

\end{document}